\section{Kan een concrete doelstelling en onderzoeksvragen voor een eigen onderzoeksonderwerp formuleren en verduidelijken}

\sectionframelogo{}

\subsection{Een BP-onderwerp}
\begin{frame}{Een onderwerp kiezen}
	
	\begin{columns}[c]

		\column{.5\textwidth}

		\begin{itemize}
			\item \textcolor{HoGentAccent3}{ICT-gerelateerd onderwerp
			\item Voldoende uitdagend, maar realistisch
			\item Concreet probleem/vraag uit het werkveld
			\item Gaat verder dan het verzamelen van informatie
			\item Nuttig voor anderen in je vakgebied
			\item Origineel en individueel (Een BP met twee gaat niet)}
		\end{itemize}
	
	
	
		\column{.5\textwidth}

	\begin{itemize}
		\item \textcolor{HoGentAccent2}{Zuivere literatuurstudie
		\item Programmeerproject
		\item Vergelijking van frameworks, producten, services zonder
		concrete case, zonder requirementsanalyse
		\item  Hergebruik resultaten stage}
	\end{itemize}
		
	\end{columns}	
	
\end{frame}


\begin{frame}{Tools tot het vinden van een onderwerp}
	\begin{enumerate}
		\item In OZT maken ze reeds een BP onderwerp
		\item De BP-gids bevat veel tips rond het zoeken naar een onderwerp
		\item Er zijn externe voorstellen op het forum van Chamilo van bedrijven
		\item \textcolor{HoGentAccent1}{Externe voorstellen van de docenten}
	\end{enumerate}
\end{frame}